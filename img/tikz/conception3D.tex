\begin{tikzpicture}
[every label/.append style={text=brown, font=\scriptsize,opacity=0.7},
 ttt/.style={font=\scriptsize}
]
		
		
		% The last trick is to cheat and use transparency
		\begin{scope}
		%	\fill[red] \firstcircle;
		%	\fill[green] \secondcircle;
		%	\fill[blue] \thirdcircle;
		\draw \firstcircle node[above left , text=brown, xshift=0.5cm] 
		{\textbf{Conception 3D web}};
		\draw \secondcircle node [above right, align=center] {
			\textbf{Architecture de }\\
			\textbf{communication}
		};
		\draw \thirdcircle node [below, align=center] {\textbf{Traçabilité et}\\\textbf{historique des données}};
		\end{scope}
		
		
	\node[ttt,brown,align=center] at (2,0.3)  {\textbf{Technos web}\\(intérop.)};
		\node[ttt,brown] at (0.7,-1.5)  {\textbf{Structure données}};
	
		\node[ttt,label={[label distance=-0.2cm]90:Surfaces implicites}] at (-1,-0.4) 
		{\cite{Grasberger2013}};
		\node[ttt,label={[label distance=-0.2cm]90:OnShape}] at (-1.5,-1.1) 
		{\cite{Baran2015}};
		\node[ttt,label={[label distance=-0.2cm]90:ChainVoxel}] at (-0.5,0.9) 
		{\cite{Imae2016}};
		
		\node[ttt,label={[label distance=-0.2cm]90:3DRepo.io}] at (0,1.5) 
		{\cite{Scully2015}};
		
	

		
\end{tikzpicture}