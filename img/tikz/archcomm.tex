	\begin{tikzpicture}
		[every label/.append style={text=Purple, font=\scriptsize,opacity=0.7},
	ttt/.style={font=\scriptsize}
	]
	
	% The last trick is to cheat and use transparency
	\begin{scope}
	%	\fill[red] \firstcircle;
	%	\fill[green] \secondcircle;
	%	\fill[blue] \thirdcircle;
	\draw \firstcircle node[above left , text=brown, xshift=0.5cm] 
	{\textbf{Conception 3D web}};
	\draw \secondcircle node [above right, align=center,text=Purple] {
		\textbf{Architecture de }\\
		\textbf{communication}
	};
	\draw \thirdcircle node [below, align=center] {\textbf{Traçabilité et}\\\textbf{historique des données}};
	\end{scope}
	
	\node[ttt,brown,align=center] at (2,0.3)  {\textbf{Technos web}\\(intérop.)};
	\node[ttt,brown] at (0.7,-1.5)  {\textbf{\tiny Structure données}};
	
	
	\node[ttt,Purple] at (2,-0.1)  {\textbf{Robustesse}};
	\node[ttt,Purple] at (3.9,-1.8)  {\textbf{Résilience}};
	\node[ttt,Purple] at (3.4,-1.5)  {\textbf{Passage à l'échelle}};
	
\node[ttt,label={[label distance=-0.2cm]90:P2P Indoor WebVR}] at (3.8,1.3) 
	{\cite{Hu2017}};
%		\node[ttt,label={[label distance=-0.2cm]90:Mobile data transfer}] at (0,-1) 
%{\cite{Stephenson2015}};
%\node[ttt,label={[label distance=-0.2cm]90:Yjs}] at (0,-1) 
%{\cite{Nicolaescu2015}};
\node[ttt,label={[label distance=-0.2cm]90:WebRTCbench}] at (5.4,1) 
{\cite{Taheri2015}};


\node[ttt,label={[label distance=-0.2cm]90:RADE}] at (4,-0.5) 
{\cite{Koskela2015}};

\node[ttt,label={[label distance=-0.2cm]90:Arch. hybride pour BIM}] at (5.3,-1) 
{\cite{Chen2014}};
	\end{tikzpicture}