%!TEX root = presentation.tex
\section{Conclusion}
\subsection{Rappel des contributions}
\begin{frame}{Résumé}

	\begin{textblock*}{95mm}(3.5cm,0.2\textheight)
	\centering \large
	\textbf{ {\color{Purple}Comment engager toutes les ressources \\ à 
			disposition 
			lors de la collaboration  sur le web?}}
\end{textblock*}

\vspace*{0.5cm}
\begin{minipage}{0.5\textwidth}
	\begin{itemize}		
		\item Architecture hybride : chaque pair participe à la distribution et au 
		stockage en temps réel et de manière égalitaire.
		\item Modèle événementiel : taxonomie et traçabilité des manipulations. 
		L'expertise des utilisateurs est sauvée et réutilisable.

	\end{itemize}
\end{minipage}\hfill
\begin{minipage}{0.5\textwidth}
	\vspace*{0.5cm}
	\centering
\inputTikZ{0.7}{img/tikz/approche12.tex}
\end{minipage}
\end{frame}


\begin{frame}{Questions de recherche}
\begin{table}
	\begin{tabular}{lcc}
		\hline
		\textbf{QR}              & \textbf{Approche orientée états} & \textbf{Approche 
			orientée événements} \\ \hline
		\textbf{QR1 Réseau}      &\tickpartial                         &  
		\tick                         \\
		\textbf{QR2 Traçabilité} &     \fail                & \tick                                  \\
		\textbf{QR3 Autonomie}   & \tickpartial                           & \tick                 \\
		\textbf{QR4 Validité}    &         \fail                         &\tick                      \\
		\textbf{QR5 Métriques}  &  \tickpartial (quali.) \fail  (quant.) & \tick (quali.)  
		\tickpartial  (quant.)       \\ \hline
	\end{tabular}
\end{table}

\end{frame}

\subsection{Perspectives}
\begin{frame}{Usages potentiels}
	La conception d'un modèle événementiel à travers l'implémentation d'une 
	plateforme 
	comme 3DEvent peut servir d'autres applications 
	asynchrones, distribuées et orientées événements.
	\begin{itemize}
		\item Application au \textbf{versionnage 3D collaboratif }avancé.
		\item \textbf{Création de scénarios} artificiels ou sur la base de traces 
		utilisateurs 
		intégrant le métier (jeux sérieux).
		\item \textbf{Traçage utilisateur et \textit{crowdsouring}} pour repérer des 
		zones d'intérêt ou proposer des résumés d'activité.
		\item Concevoir des \textbf{audits et  des outils de surveillance} pour les 
		données 
		3D issues de la collaboration.
		
	\end{itemize}
\end{frame}




\begin{frame}{Perspectives}
	Comparaison quantitative des deux approches (mêmes 
	interface/scénario/réseau)
	
	Virtualisation des comportements collaboratifs \cite{Haque2016}\cite{Erb2014}
	
	Gestion des conflits (super n\oe uds)
	\cite{Klamer2013a} % Conflict resolution event sourcing 
	\cite{Baldoni2007}
	
	
	Compression 3D \cite{Maglo2013a} (avec événements) 
\end{frame}